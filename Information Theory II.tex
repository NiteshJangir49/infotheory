\documentclass[11pt]{article}

%%% XeLaTeX Font Definitions

\usepackage{titlesec}
\usepackage{titling}
\usepackage{xunicode}
\usepackage{fontspec,xltxtra,xunicode}
\usepackage[table,xcdraw]{xcolor}
\defaultfontfeatures{Mapping=tex-text}
\usepackage{bigints}
\usepackage{booktabs}
\usepackage{bm}

% Uncomment below to change default features 
%\setromanfont[Mapping=tex-text]{Hoefler Text}
%\setsansfont[Scale=MatchLowercase,Mapping=tex-text]{Gill Sans}
%\setmonofont[Scale=MatchLowercase]{Andale Mono}

% Specify different font for section headings
\newfontfamily\headingfont[]{Lucida Grande Bold}
\newfontfamily\titlefont[]{Optima}

\titleformat*{\section}{\Large\headingfont}
\titleformat*{\subsection}{\large\headingfont}
\titleformat*{\subsubsection}{\large\headingfont}
\renewcommand{\maketitlehooka}{\titlefont}

%%% Remove the "abstract" word before the abstract

\newcommand{\overbar}[1]{\mkern 1.5mu\overline{\mkern-1.5mu#1\mkern-1.5mu}\mkern 1.5mu}

\usepackage{abstract}
\renewcommand{\abstractname}{}    % clear the title
\renewcommand{\absnamepos}{empty} % originally center

%%% Actual Preamble

%\headheight=8pt
%\topmargin=3pt
%\textheight=624pt
%\textwidth=432pt
%\oddsidemargin=18pt
%\evensidemargin=18pt
\usepackage{amsmath}
\usepackage{amsfonts}
\usepackage{amssymb}
\usepackage{amsthm}
\usepackage{comment}
\usepackage{epsfig}
\usepackage{psfrag}
\usepackage{mathtools}
\DeclarePairedDelimiter{\ceil}{\lceil}{\rceil}

%\usepackage{sseq} (if you need to draw spectral sequences, please use this package, available at http://wwwmath.uni-muenster.de/u/tbauer/)
\usepackage{mathrsfs}
\usepackage{amscd}
\usepackage[all]{xy}
\usepackage{rotating}
\usepackage{lscape}
\usepackage{amsbsy}
\usepackage{verbatim}
\usepackage{moreverb}
\usepackage{mathdots}
\usepackage{setspace}
%\usepackage{eucal}
\usepackage{hyperref}
\usepackage{pgfplots}%http://www.ctan.org/pkg/pgfplots

\usepackage{listings}
\usepackage[margin=1in]{geometry}
\pagestyle{plain}
\theoremstyle{definition}
\newtheorem{theorem}{Theorem}%[section]
\newtheorem{prop}{Proposition}
\newtheorem{lemma}{Lemma}
\newtheorem{corollary}[theorem]{Corollary}
%\theoremstyle{definition}
\newtheorem{definition}{Definition}
\newtheorem{notation}{Notation}
\newtheorem{summary}{Summary}
\newtheorem{note}{Note}
\newtheorem{construction}[theorem]{Construction}
%\theoremstyle{remark}
\newtheorem{remark}{Remark}
\newtheorem{example}{Example}
\newtheorem{question}[example]{Question}
\DeclareMathOperator{\Aut}{Aut}
\DeclareMathOperator{\coeq}{coeq}
\DeclareMathOperator{\colim}{colim}
\DeclareMathOperator{\cone}{cone}
\DeclareMathOperator{\Der}{Der}
\DeclareMathOperator{\Ext}{Ext}
\DeclareMathOperator{\hocolim}{hocolim}
\DeclareMathOperator{\holim}{holim}
\DeclareMathOperator{\Hom}{Hom}
\DeclareMathOperator{\Iso}{Iso}
\DeclareMathOperator{\Map}{Map}
\DeclareMathOperator{\Tot}{Tot}
\DeclareMathOperator{\Tor}{Tor}
\DeclareMathOperator{\Spec}{Spec}
\newcommand{\TMF}{\mathit{TMF}}
\newcommand{\tmf}{\mathit{tmf}}
\newcommand{\Mell}{\mathcal M_{\mathit{ell}}}
\newcommand{\Mord}{\mathcal M_{\mathit{ell}}^{\mathit{ord}}}
\newcommand{\Mss}{\mathcal M_{\mathit{ell}}^{\mathit{ss}}}
\newcommand{\Mbar}{\overline{\mathcal M}_{\mathit{ell}}}
\newcommand{\Mfg}{\mathcal M_{\mathit{FGL}}}
\newcommand{\MU}{\mathit{MU}}
\newcommand{\MP}{\mathit{MP}}
\newcommand{\Lk}{L_{K(n)}}
\newcommand{\Lone}{L_{K(1)}}
\newcommand{\Ltwo}{L_{K(2)}}
\newcommand{\Sp}{\mathbf{Sp}}
\newcommand{\Eoo}{E_\infty}
\newcommand{\Aoo}{A_\infty}
\newcommand{\CP}{\mathbb{CP}^\infty}
\newcommand{\GL}{\mathit{GL}}
\newcommand{\gl}{\mathit{gl}}
\newcommand{\nn}{\nonumber}
\newcommand{\nid}{\noindent}
\newcommand{\ra}{\rightarrow}
\newcommand{\la}{\leftarrow}
\newcommand{\xra}{\xrightarrow}
\newcommand{\xla}{\xleftarrow}
\newcommand{\weq}{\xrightarrow{\sim}}
\newcommand{\cofib}{\rightarrowtail}
\newcommand{\fib}{\twoheadrightarrow}
 \newcommand{\xhdr}[1]{\vspace{2mm}\noindent{{\bf #1.}}}

\def\llarrow{   \hspace{.05cm}\mbox{\,\put(0,-2){$\leftarrow$}\put(0,2){$\leftarrow$}\hspace{.45cm}}}
\def\rrarrow{   \hspace{.05cm}\mbox{\,\put(0,-2){$\rightarrow$}\put(0,2){$\rightarrow$}\hspace{.45cm}}}
\def\lllarrow{  \hspace{.05cm}\mbox{\,\put(0,-3){$\leftarrow$}\put(0,1){$\leftarrow$}\put(0,5){$\leftarrow$}\hspace{.45cm}}}
\def\rrrarrow{  \hspace{.05cm}\mbox{\,\put(0,-3){$\rightarrow$}\put(0,1){$\rightarrow$}\put(0,5){$\rightarrow$}\hspace{.45cm}}}
\def\cA{\mathcal A}\def\cB{\mathcal B}\def\cc{\mathbf C}\def\cd{\mathbf D}
\def\ce{\mathcal E}\def\cf{\mathcal F}\def\cG{\mathcal G}\def\cH{\mathcal H}
\def\cI{\mathcal I}\def\cJ{\mathcal J}\def\cK{\mathcal K}\def\cL{\mathcal L}
\def\cM{\mathbf M}\def\cN{\mathcal N}\def\cO{\mathbf O}\def\cP{\mathcal P}
\def\cQ{\mathcal Q}\def\cR{\mathcal R}\def\cS{\mathcal S}\def\cT{\mathcal T}
\def\cU{\mathcal U}\def\cV{\mathcal V}\def\cW{\mathcal W}\def\cX{\mathcal X}
\def\cY{\mathcal Y}\def\cZ{\mathcal Z}
\def\AA{\mathbb A}\def\BB{\mathbb B}\def\CC{\mathbb C}\def\DD{\mathbb D}
\def\EE{\mathbb E}\def\FF{\mathbb F}\def\GG{\mathbb G}\def\HH{\mathbb H}
\def\II{\mathbb I}\def\JJ{\mathbb J}\def\KK{\mathbb K}\def\LL{\mathbb L}
\def\MM{\mathbb M}\def\NN{\mathbb N}\def\OO{\mathbb O}\def\PP{\mathbb P}
\def\QQ{\mathbb Q}\def\RR{\mathbb R}\def\SS{\mathbb S}\def\TT{\mathbb T}
\def\UU{\mathbb U}\def\VV{\mathbb V}\def\WW{\mathbb W}\def\XX{\mathbb X}
\def\YY{\mathbb Y}\def\ZZ{\mathbb Z}

\newcommand{\MFGL}{\mathcal M_{\mathit{FGL}}}
\newcommand{\calO}{{\mathcal O}}
\newcommand{\calC}{{\mathcal C}}
\newcommand{\set}{{\mathrm{Set}}}
\newcommand{\Deltab}{{\mathbf \Delta}}
\newcommand{\spet}{\mathrm{Spec}^\mathrm{\acute{e}t}}
\newcommand{\Z}{\mathbb Z}
\DeclareMathOperator{\Spf}{Spf}

\usepackage{fancyhdr}
\setlength{\headheight}{15.2pt}
\pagestyle{fancy}

\lhead{2016-17}
\chead{Information Theory}
\rhead{Manan Shah}

\graphicspath{{./figures/}}

\begin{document}
\title{\headingfont{Information Theory, Part II}}
\author{Manan Shah\\ \texttt{manan.shah.777@gmail.com} \\ The Harker School}
\maketitle
\begin{abstract}
This document contains lecture notes from Harker's Advanced Topics in Mathematics class in Information Theory II, taught by Dr. Anuradha Aiyer. This course is the second part of a two part offering that explores the basic concepts of Information Theory, as initially described by Claude Elwood Shannon at Bell Labs in 1948. In Part 2 of the course, we explore other applications of Information Theory to the disciplines of Gambling, Statistics, Physics, Computer science, Economics and Philosophy. These notes were taken using TeXShop and \LaTeX2$\epsilon$ and will be updated for each class. The reader is advised to note any errata at the source control repository \texttt{https://github.com/mananshah99/infotheory}.
\end{abstract}
\tableofcontents
\newpage

%% Notes start here

\section{Unit 1: Gambling}

We'll discuss the duality between the growth rate of investment (i.e. a horse race) and the entropy rate of the horse race and how the side information's financial value is tied to mutual information. 

\definition[Horse Race] We have $m$ horses in a race in which the $i$th horse wins with probability $p_i$. If horse $i$ wins, the payoff is $o_i$ for 1\footnote{There are two ways to describe a bet: either $a$ for 1 or $b$ to 1. The first notation indicates an exchange that happens prior to the race, and the latter indicates and exchange that happens post-race (although in both cases the horses are picked before the race). More concretely, $a$ for 1 indicates that if one places \$1 on a particular horse before the race, the payoff is \$a iff the horse wins and \$0 if the horse loses. $b$ to 1 indicates that one would pay \$1 after the race if a particular horse loses and win \$b if the horse wins. The equivalency between these scenarios is $b = a-1$.}. We'll assume that the gambler invests his wealth across all horses and doesn't hold on to any of his money. Specifically, $b_i$ is the fraction of wealth invested in horse $i$ where $b_i \geq 0$ and $\sum b_i = 1$. If horse $i$ wins, the gambler wins $o_i b_i$; this case occurs with probability $p_i$. The wealth at the end of the race is a random variable which we will attempt to maximize. 

\subsection{Repeated Gambling}

Define $S_n$ as the total growth in the gambler's wealth after $n$ races. We have $$S_n = \prod_{j=1}^n S(X_j) \qquad \text{and} \qquad S(X_j) = b(X_j) o(X_j)$$with $X$ representing the horse that wins (this changes between races). Here, $S(X_i)$ represents the factor by which the gambler's wealth grows. We can define the doubling rate of a race $W$ as $$E(\log S(X)) = \sum_{k=1}^m p_k \log (b_k o_k) = W(b, p)$$

\theorem Let race outcomes $X_1, X_2, X_3, \dots, X_n$ be identically and independently distributed $\sim p(x)$. The wealth of a gambler using betting strategy $b$ grows exponentially at the rate $W(b, p)$ such that $S_n = 2^{n W(b, p)}$.

\begin{proof}
Functions of independent random variables are also independent, so $\log S(X_1), \dots \log S(X_n)$ are i.i.d. From our earlier definition of $S_n$ we have $$\frac{1}{n} \log S_n = \frac{1}{n} \sum \log S(X_i)$$By the weak law of large numbers\footnote{See \texttt{http://mathworld.wolfram.com/WeakLawofLargeNumbers.html} for more information.}, this equates to $E(\log S(X)) = W(b, p)$. So we can conclude that $S_n =  2^{n W(b, p)}$ and the proof is complete. So if to maximize $S_n$, we'll need to maximize $W$. 
\end{proof}
\definition The optimum doubling rate over all choices of $b_i$ is $$W^*(p) = \max_b W(b, p) = \max_{b: \: b_i \geq 0, \: \sum b_i = 1} \sum_i p_i \log b_i o_i$$We must formally maximize $W(b, p)$ such that $\sum b_i = 1$. To do this, we'll apply Lagrange optimization. We have $$J(b) = \sum p_i \log b_i o_i + \lambda \sum b_i$$Taking the partial with respect to $b_i$ and setting it equal to 0, $$\frac{\partial J}{\partial b_i} = \frac{p_i}{b_i} + \lambda$$ where $i \in \{ 1 \dots m \}$. $\sum b_i = 1$ results in $b_i = p_i$. Technically, we'd have to take the second derivative to prove that this is a maximum; this verification is left to the reader. 

\theorem $W^* = \sum p_i \log o_i - H(p)$

\begin{proof}
\begin{align*}
W(b, p) &= \sum p_i \log b_i o_i  \\
&= p_i \log \left( \frac{b_i}{p_i} \times p_i o_i \right) \\
&= \sum p_i \log o_i - H(p) - D(p || b)
\end{align*}
The last term, $D$, is known as relative entropy. It has some of the same properties of entropy, one of them being that $D \geq 0$. So, we have that $W(b, p) \leq \sum p_i \log o_i - H(p)$ with equality when $p = b$. 
\end{proof}

The function $D$, known as the relative entropy or Kullback-Liebler Divergence, is a measure of distance\footnote{This isn't technically a measure of distance as it doesn't satisfy the triangle inequality} between two distributions. If $p$ and $q$ are the two distributions, then $D(p || q)$ is a measure of inefficiency of assuming $q$ when the true distribution is $p$. The average code length for distribution $p$ is $H(p)$, but if we were to use the code for $q$ to encode $p$, then $H(p) + D(p||q)$ bits. 

\definition The KL divergence $D(p||q)$ is expressed as $$D(p||q) = \sum_x p(x) \log \frac{p(x)}{q(x)} = E_x \left[ \log \frac{p(x)}{q(x)} \right] $$ where $0 \log 0/q = 0$ and $p \log p / 0 = \infty$. We can then write $I(X; Y) = \sum \sum p(x, y) \log \frac{p(x, y)}{p(x) p(y)}$ which is simplified to $D(p(x, y) || p(x) p(Y))$

\example $p(0) = 1-r, q(0) = 1-s, p(1) = r, q(1) = s$
$$D(p||q) = (1-r) \log \frac{1-r}{1-s} + r \log \frac{r}{s}$$ 
$$D(q||p) = (1-s) \log \frac{1-s}{1-r} + s \log \frac{s}{r}$$

\example Consider a case with two horses where horse 1 wins with probability $p_1$ and horse 2 wins with $p_2$. Assume even odds (2-for-1). (a) What is the optimal bet? (b) Doubling rate? (c) Resulting wealth? 

(a) The optimal bet is according to the probabilities of the horses, (b) The doubling rate is $1-H(p)$, and the resulting wealth (c) is $2^{n(1-H(p)}$

We further have that $W(b, p) = \sum p \log \frac{p_i}{r_i} - \sum p \log \frac{p}{b} = D(p || r) - D(p || b)$. The doubling rate is the difference between the distance of the bookie's estimates from the truth. The gambler only makes money when $b$ is closer than $r$. When the odds are $m$-for-1, we have $$W^*(p) = D(p || 1/m)= \log m - H(p)$$and $W^*(p) + H(p) = \log m$.

\newpage
\section*{Appendix A---Quotes}
\begin{itemize}
\item ``I have a problem. It's called gambling.'' (Dr. Aiyer)
\item ``What's $E$? Entropy?'' (David Zhu)
\item ``So is it the strong law of weak numbers?'' (Jerry Chen)
\item ``It's like a half life... but it's a double life'' (Steven Cao)
\item ``Isn't this just Lagrange?''  \\ 10 minutes later.. \\ ``Wait, how do you do Lagrange again?'' (Swapnil Garg)
\end{itemize}
\end{document}
