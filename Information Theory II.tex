\documentclass[11pt]{article}

%%% XeLaTeX Font Definitions

\usepackage{titlesec}
\usepackage{titling}
\usepackage{xunicode}
\usepackage{fontspec,xltxtra,xunicode}
\usepackage[table,xcdraw]{xcolor}
\defaultfontfeatures{Mapping=tex-text}
\usepackage{bigints}
\usepackage{booktabs}
\usepackage{bm}

% Uncomment below to change default features 
%\setromanfont[Mapping=tex-text]{Hoefler Text}
%\setsansfont[Scale=MatchLowercase,Mapping=tex-text]{Gill Sans}
%\setmonofont[Scale=MatchLowercase]{Andale Mono}

% Specify different font for section headings
\newfontfamily\headingfont[]{Lucida Grande Bold}
\newfontfamily\titlefont[]{Optima}

\titleformat*{\section}{\Large\headingfont}
\titleformat*{\subsection}{\large\headingfont}
\titleformat*{\subsubsection}{\large\headingfont}
\renewcommand{\maketitlehooka}{\titlefont}

%%% Remove the "abstract" word before the abstract

\newcommand{\overbar}[1]{\mkern 1.5mu\overline{\mkern-1.5mu#1\mkern-1.5mu}\mkern 1.5mu}

\usepackage{abstract}
\renewcommand{\abstractname}{}    % clear the title
\renewcommand{\absnamepos}{empty} % originally center

%%% Actual Preamble

%\headheight=8pt
%\topmargin=3pt
%\textheight=624pt
%\textwidth=432pt
%\oddsidemargin=18pt
%\evensidemargin=18pt
\usepackage{amsmath}
\usepackage{amsfonts}
\usepackage{amssymb}
\usepackage{amsthm}
\usepackage{comment}
\usepackage{epsfig}
\usepackage{psfrag}
\usepackage{mathtools}

\DeclarePairedDelimiter{\ceil}{\lceil}{\rceil}

%\usepackage{sseq} (if you need to draw spectral sequences, please use this package, available at http://wwwmath.uni-muenster.de/u/tbauer/)
\usepackage{mathrsfs}
\usepackage{amscd}
\usepackage[all]{xy}
\usepackage{rotating}
\usepackage{lscape}
\usepackage{amsbsy}
\usepackage{verbatim}
\usepackage{moreverb}
\usepackage{mathdots}
\usepackage{setspace}
%\usepackage{eucal}
\usepackage{hyperref}
\usepackage{pgfplots}%http://www.ctan.org/pkg/pgfplots

\usepackage{listings}
\usepackage[margin=1in]{geometry}
\pagestyle{plain}
\theoremstyle{definition}
\newtheorem{theorem}{Theorem}%[section]
\newtheorem{prop}{Proposition}
\newtheorem{lemma}{Lemma}
\newtheorem{corollary}[theorem]{Corollary}
%\theoremstyle{definition}
\newtheorem{definition}{Definition}
\newtheorem{notation}{Notation}
\newtheorem{summary}{Summary}
\newtheorem{note}{Note}
\newtheorem{construction}[theorem]{Construction}
%\theoremstyle{remark}
\newtheorem{remark}{Remark}
\newtheorem{example}{Example}
\newtheorem{question}[example]{Question}
\DeclareMathOperator{\Aut}{Aut}
\DeclareMathOperator{\coeq}{coeq}
\DeclareMathOperator{\colim}{colim}
\DeclareMathOperator{\cone}{cone}
\DeclareMathOperator{\Der}{Der}
\DeclareMathOperator{\Ext}{Ext}
\DeclareMathOperator{\hocolim}{hocolim}
\DeclareMathOperator{\holim}{holim}
\DeclareMathOperator{\Hom}{Hom}
\DeclareMathOperator{\Iso}{Iso}
\DeclareMathOperator{\Map}{Map}
\DeclareMathOperator{\Tot}{Tot}
\DeclareMathOperator{\Tor}{Tor}
\DeclareMathOperator{\Spec}{Spec}
\newcommand{\TMF}{\mathit{TMF}}
\newcommand{\tmf}{\mathit{tmf}}
\newcommand{\Mell}{\mathcal M_{\mathit{ell}}}
\newcommand{\Mord}{\mathcal M_{\mathit{ell}}^{\mathit{ord}}}
\newcommand{\Mss}{\mathcal M_{\mathit{ell}}^{\mathit{ss}}}
\newcommand{\Mbar}{\overline{\mathcal M}_{\mathit{ell}}}
\newcommand{\Mfg}{\mathcal M_{\mathit{FGL}}}
\newcommand{\MU}{\mathit{MU}}
\newcommand{\MP}{\mathit{MP}}
\newcommand{\Lk}{L_{K(n)}}
\newcommand{\Lone}{L_{K(1)}}
\newcommand{\Ltwo}{L_{K(2)}}
\newcommand{\Sp}{\mathbf{Sp}}
\newcommand{\Eoo}{E_\infty}
\newcommand{\Aoo}{A_\infty}
\newcommand{\CP}{\mathbb{CP}^\infty}
\newcommand{\GL}{\mathit{GL}}
\newcommand{\gl}{\mathit{gl}}
\newcommand{\nn}{\nonumber}
\newcommand{\nid}{\noindent}
\newcommand{\ra}{\rightarrow}
\newcommand{\la}{\leftarrow}
\newcommand{\xra}{\xrightarrow}
\newcommand{\xla}{\xleftarrow}
\newcommand{\weq}{\xrightarrow{\sim}}
\newcommand{\cofib}{\rightarrowtail}
\newcommand{\fib}{\twoheadrightarrow}
 \newcommand{\xhdr}[1]{\vspace{2mm}\noindent{{\bf #1.}}}

\def\llarrow{   \hspace{.05cm}\mbox{\,\put(0,-2){$\leftarrow$}\put(0,2){$\leftarrow$}\hspace{.45cm}}}
\def\rrarrow{   \hspace{.05cm}\mbox{\,\put(0,-2){$\rightarrow$}\put(0,2){$\rightarrow$}\hspace{.45cm}}}
\def\lllarrow{  \hspace{.05cm}\mbox{\,\put(0,-3){$\leftarrow$}\put(0,1){$\leftarrow$}\put(0,5){$\leftarrow$}\hspace{.45cm}}}
\def\rrrarrow{  \hspace{.05cm}\mbox{\,\put(0,-3){$\rightarrow$}\put(0,1){$\rightarrow$}\put(0,5){$\rightarrow$}\hspace{.45cm}}}
\def\cA{\mathcal A}\def\cB{\mathcal B}\def\cc{\mathbf C}\def\cd{\mathbf D}
\def\ce{\mathcal E}\def\cf{\mathcal F}\def\cG{\mathcal G}\def\cH{\mathcal H}
\def\cI{\mathcal I}\def\cJ{\mathcal J}\def\cK{\mathcal K}\def\cL{\mathcal L}
\def\cM{\mathbf M}\def\cN{\mathcal N}\def\cO{\mathbf O}\def\cP{\mathcal P}
\def\cQ{\mathcal Q}\def\cR{\mathcal R}\def\cS{\mathcal S}\def\cT{\mathcal T}
\def\cU{\mathcal U}\def\cV{\mathcal V}\def\cW{\mathcal W}\def\cX{\mathcal X}
\def\cY{\mathcal Y}\def\cZ{\mathcal Z}
\def\AA{\mathbb A}\def\BB{\mathbb B}\def\CC{\mathbb C}\def\DD{\mathbb D}
\def\EE{\mathbb E}\def\FF{\mathbb F}\def\GG{\mathbb G}\def\HH{\mathbb H}
\def\II{\mathbb I}\def\JJ{\mathbb J}\def\KK{\mathbb K}\def\LL{\mathbb L}
\def\MM{\mathbb M}\def\NN{\mathbb N}\def\OO{\mathbb O}\def\PP{\mathbb P}
\def\QQ{\mathbb Q}\def\RR{\mathbb R}\def\SS{\mathbb S}\def\TT{\mathbb T}
\def\UU{\mathbb U}\def\VV{\mathbb V}\def\WW{\mathbb W}\def\XX{\mathbb X}
\def\YY{\mathbb Y}\def\ZZ{\mathbb Z}

\newcommand{\MFGL}{\mathcal M_{\mathit{FGL}}}
\newcommand{\calO}{{\mathcal O}}
\newcommand{\calC}{{\mathcal C}}
\newcommand{\set}{{\mathrm{Set}}}
\newcommand{\Deltab}{{\mathbf \Delta}}
\newcommand{\spet}{\mathrm{Spec}^\mathrm{\acute{e}t}}
\newcommand{\Z}{\mathbb Z}
\DeclareMathOperator{\Spf}{Spf}

\usepackage{fancyhdr}
\setlength{\headheight}{15.2pt}
\pagestyle{fancy}

\usepackage{booktabs}

\lhead{2016-17}
\chead{Information Theory}
\rhead{Manan Shah}

\graphicspath{{./figures/}}

\begin{document}
\title{\headingfont{Information Theory, Part II}}
\author{Manan Shah\\ \texttt{manan.shah.777@gmail.com} \\ The Harker School}
\maketitle
\begin{abstract}
This document contains lecture notes from Harker's Advanced Topics in Mathematics class in Information Theory II, taught by Dr. Anuradha Aiyer. This course is the second part of a two part offering that explores the basic concepts of Information Theory, as initially described by Claude Elwood Shannon at Bell Labs in 1948. These notes were taken using TeXShop and \LaTeX2$\epsilon$ and will be updated for each class. The reader is advised to note any errata at the source control repository \texttt{https://github.com/mananshah99/infotheory}.
\end{abstract}
\tableofcontents
\newpage

%% Notes start here

\section{Unit 1: Gambling}

We'll discuss the duality between the growth rate of investment (i.e. a horse race) and the entropy rate of the horse race and how the side information's financial value is tied to mutual information. 

\definition[Horse Race] We have $m$ horses in a race in which the $i$th horse wins with probability $p_i$. If horse $i$ wins, the payoff is $o_i$ for 1\footnote{There are two ways to describe a bet: either $a$ for 1 or $b$ to 1. The first notation indicates an exchange that happens prior to the race, and the latter indicates and exchange that happens post-race (although in both cases the horses are picked before the race). More concretely, $a$ for 1 indicates that if one places \$1 on a particular horse before the race, the payoff is \$a iff the horse wins and \$0 if the horse loses. $b$ to 1 indicates that one would pay \$1 after the race if a particular horse loses and win \$b if the horse wins. The equivalency between these scenarios is $b = a-1$.}. We'll assume that the gambler invests his wealth across all horses and doesn't hold on to any of his money. Specifically, $b_i$ is the fraction of wealth invested in horse $i$ where $b_i \geq 0$ and $\sum b_i = 1$. If horse $i$ wins, the gambler wins $o_i b_i$; this case occurs with probability $p_i$. The wealth at the end of the race is a random variable which we will attempt to maximize. 

\subsection{Repeated Gambling}

Define $S_n$ as the total growth in the gambler's wealth after $n$ races. We have $$S_n = \prod_{j=1}^n S(X_j) \qquad \text{and} \qquad S(X_j) = b(X_j) o(X_j)$$with $X$ representing the horse that wins (this changes between races). Here, $S(X_i)$ represents the factor by which the gambler's wealth grows. We can define the doubling rate of a race $W$ as $$E(\log S(X)) = \sum_{k=1}^m p_k \log (b_k o_k) = W(b, p)$$

\theorem Let race outcomes $X_1, X_2, X_3, \dots, X_n$ be identically and independently distributed $\sim p(x)$. The wealth of a gambler using betting strategy $b$ grows exponentially at the rate $W(b, p)$ such that $S_n = 2^{n W(b, p)}$.

\begin{proof}
Functions of independent random variables are also independent, so $\log S(X_1), \dots \log S(X_n)$ are i.i.d. From our earlier definition of $S_n$ we have $$\frac{1}{n} \log S_n = \frac{1}{n} \sum \log S(X_i)$$By the weak law of large numbers\footnote{See \texttt{http://mathworld.wolfram.com/WeakLawofLargeNumbers.html} for more information.}, this equates to $E(\log S(X)) = W(b, p)$. So we can conclude that $S_n =  2^{n W(b, p)}$ and the proof is complete. So if to maximize $S_n$, we'll need to maximize $W$. 
\end{proof}
\definition The optimum doubling rate over all choices of $b_i$ is $$W^*(p) = \max_b W(b, p) = \max_{b: \: b_i \geq 0, \: \sum b_i = 1} \sum_i p_i \log b_i o_i$$We must formally maximize $W(b, p)$ such that $\sum b_i = 1$. To do this, we'll apply Lagrange optimization. We have $$J(b) = \sum p_i \log b_i o_i + \lambda \sum b_i$$Taking the partial with respect to $b_i$ and setting it equal to 0, $$\frac{\partial J}{\partial b_i} = \frac{p_i}{b_i} + \lambda$$ where $i \in \{ 1 \dots m \}$. Solving for $b_i$ as a function of $p_i$ and $\lambda$, substituting the resulting value into the constraint $\sum b_i = 1$, and evaluating the differential expression with  $\lambda = -1$ results in $b_i = p_i$. Technically, we'd have to take the second derivative to prove that this is a maximum; this verification is left to the reader. 

\theorem $W^* = \sum p_i \log o_i - H(p)$

\begin{proof}
\begin{align*}
W(b, p) &= \sum p_i \log b_i o_i  \\
&= p_i \log \left( \frac{b_i}{p_i} \times p_i o_i \right) \\
&= \sum p_i \log o_i - H(p) - D(p || b)
\end{align*}
The last term, $D$, is known as relative entropy. It has some of the same properties of entropy, one of them being that $D \geq 0$. So,  $W(b, p) \leq \sum p_i \log o_i - H(p)$ with equality when $p = b$. 
\end{proof}

\subsection{Kullback-Liebler Divergence}

The function $D$, known as the relative entropy or Kullback-Liebler Divergence, is a measure of distance\footnote{This isn't technically a measure of distance as it doesn't satisfy the triangle inequality} between two distributions. If $p$ and $q$ are the two distributions, then $D(p || q)$ is a measure of inefficiency of assuming $q$ when the true distribution is $p$. The average code length for distribution $p$ is $H(p)$, but if we were to use the code for $q$ to encode $p$, then $H(p) + D(p||q)$ bits. 

\definition[Kullback-Liebler Divergence] The KL divergence $D(p||q)$ is expressed as $$D(p||q) = \sum_x p(x) \log \frac{p(x)}{q(x)} = E_x \left[ \log \frac{p(x)}{q(x)} \right] $$ where $0 \log 0/q = 0$ and $p \log p / 0 = \infty$. We can then write $I(X; Y) = \sum \sum p(x, y) \log \frac{p(x, y)}{p(x) p(y)}$ which is simplified to $D(p(x, y) || p(x) p(Y))$

\example $p(0) = 1-r, q(0) = 1-s, p(1) = r, q(1) = s$
$$D(p||q) = (1-r) \log \frac{1-r}{1-s} + r \log \frac{r}{s}$$ 
$$D(q||p) = (1-s) \log \frac{1-s}{1-r} + s \log \frac{s}{r}$$

\example Consider a case with two horses where horse 1 wins with probability $p_1$ and horse 2 wins with $p_2$. Assume even odds (2-for-1). (a) What is the optimal bet? (b) Doubling rate? (c) Resulting wealth? 

(a) The optimal bet is according to the probabilities of the horses, (b) The doubling rate is $1-H(p)$, and the resulting wealth (c) is $2^{n(1-H(p))}$

We further have that $W(b, p) = \sum p \log \frac{p_i}{r_i} - \sum p \log \frac{p}{b} = D(p || r) - D(p || b)$ where $r_i = 1/o_i$. The doubling rate is the difference between the distance of the bookie's estimates from the truth. The gambler only makes money when $b$ is closer than $r$. When the odds are $m$-for-1, we have $$W^*(p) = D(p || 1/m)= \log m - H(p)$$and $W^*(p) + H(p) = \log m$.
 
 \example Three horses run a race. A gambler offers 3-for-1 odds on each horse. Fair odds under the assumption that all horses are equally likely to win. $p = (1/2, 1/4, 1/4)$. (a) Expected wealth, (b) $b^*$, (c) $W^*$
 
(a) We have that $W(b) = \sum p_i \log b_i o_i = \sum p_i \log 3b$ since $o_i = 1/3$ due to fair odds. Therefore, $W(b) = \sum p_i \log 3 + \sum p_i \log b_i = \log 3 + \sum p_i \log b_i$. (b) $b^* = p = (1/2, 1/4/, 1/4)$ and (c) $W^* = W(b^*) - \log 3 - 3/2$. Note that we can solve (c) with the identity discussed above. 

\subsection{The Value of Side Information}

One measure is the increase in the doubling rate based on the information. We'll connect this increase with mutual information (as we connected $W^*$ with KL divergence and entropy before). Define $X \in \{1, 2, \dots m \}$ as the horse betting space, $p(x)$ as the probabilities associated with $1 \rightarrow m$, $o(x)$ for 1 odds, and $y$ as the side information. Furthermore, we have $\sum_x b(x|y)$ as the conditional betting depending on side information $y$ and $b(x|y)$ as the proportion of wealth bet on horse $x$ when $y$ is observed. Based on these definitions, we have $$W^*(X) = \max_{b(x)} \sum_x p(x) \log b(x) o(x)$$and given our side information, $$W^*(X|Y) = \max_{b(x|y)} \sum_{x, y} p(x, y) \log b(x|y) o(x)$$so we have $$\Delta W = W^*(X|Y) - W(X)$$

\theorem The increase doubling rate $\Delta W$ due to side information $Y$ for a horse race $X$ is $\Delta W = I(X; Y).$

\begin{proof}
We have that $b^*(x|y) = p(x|y)$. Since $W^*(X|Y) = \max_{b(x|y)} E(\log S)$\footnote{$S$ was defined earlier as the aggregate wealth}. This equates to $\max_{b(x|y)} \sum p(x, y) \log [o(x) b(x|y)]$. So, we have that $$W^*(X|Y) = \sum p(x,y) \log [p(x) p(x|y)] = \sum p(x) \log o(x) - H(X|Y)$$Without side information $W^* = \sum p(x) \log o(x) - H(X)$, so we have $\Delta W = \sum p(x) \log o(x) - H(X|Y) - [\sum p(x) \log o(x) - H(X)]$. Finally, we have $\Delta W = H(X) - H(X|Y) = I(X;Y)$.
\end{proof}

\example Given a three horse race $p = (1/2, 1/4, 1/4)$ with odds with respect to the false distribution $r_1, r_2, r_3 = (1/4, 1/4, 1/2)$ and $o_1, o_2, o_3 = (4, 4, 2)$\footnote{This is because $o_i = 1/r_i$ when determining the odds given the false distribution. ``Fair odds'' are defined such that $\sum 1/o_i = 1$}. Find (a) the entropy of the race and (b) $(b_1, b_2, b_3)$ such that compounded wealth $\rightarrow \infty$. 

The entropy of the race is easily calculated as $3/2$. It's intuitive that $b_i = o_i p_i$ so we have $(2, 1, 1/2)$, which we re-normalize to $(4/7, 2/7, 1/7)$. Our final $W = \sum p_i \log b_i o_i$. 

\example Let the distribution be $(p_1, p_2, p_3)$ with odds $o = (1, 1, 1)$ and wealth proportions $b = (b_1, b_2, b_3)$. $S_n \rightarrow 0$ exponentially. (a) Find the exponent, (b) $b^*$, and (c) What $p$ causes $S_n \rightarrow 0$ at the fastest rate. 

We always have that $b_i = \frac{p_i o_i}{\sum_i b_i}$, so we can write $b^* = p$. Furthermore, the exponent is simply the doubling rate $W = \sum p_i \log b_i o_i$, and the $P$ that causes $S_n \rightarrow 0$ most quickly is the one that maximizes $H(p)$ or $p = (1/3, 1/3, 1/3)$. 

\example Now assume you have the most common form of side information, the past performance of the horses. If the results of each successive horse race are independent, then this additional information will not reveal anything new about the race. Suppose instead that the races are dependent. How might we calculate $W^*(X_k | X_{k-1}, X_{k-2},..., X_1)$?\\

\noindent At the end of of $n$ races, $$S_n = \prod_i S(X_i)$$
Therefore,
\begin{align*}
\frac{1}{n} \mathrm{E}\left[\log{S_n}\right] &= \frac{1}{n} \sum_{i=1}^{n} \mathrm{E}\left[\log{S(X_i)}\right]\\
							        &= \frac{1}{n} \sum_{i=1}^{n} \left[\log{m} - H(X_i | ...)\right]\\
							  W^*&=  \log{m} - \frac{H(X_1, X_2,...,X_n)}{n}
\end{align*}
The term $\frac{H(X_1, X_2,...,X_n)}{n}$ can be thought of as the average entropy across all the races. 

\section{Unit 2: Statistics}

\subsection{Introduction}
Statistics considers two types of studies: observational and experimental. In an experimental study, treatments are assigned to subjects; this is not the case in observational studies. The first part of this topic was covered with traditional statistics worksheets involving the definition of $p$-values and statistical tests ($t$, $z$, etc.) A diagram that connects these concepts is as follows:

\begin{table}[ht]
\centering
\begin{tabular}{@{}lll@{}}
\toprule
Information Theory & Statistics            & Machine Learning      \\ \midrule
Source             & Observational Studies & Unsupervised Learning \\
Channel            & Experimental Studies  & Supervised Learning   \\ \bottomrule
\end{tabular}
\end{table}

We'll start by modeling the source, which has a distribution $Q$ and outputs the vector $X$. We will define a notion of $X$ as too ridiculous to have come from $Q$. It is incorrect to define the ridiculousness criterion as ``$\text{Prob}_Q(X)$ is small'' as looking at one event whose probability will almost always be small is insufficient.

\definition $X$ is ``too ridiculous'' to have originated from source $Q$ if and only if probability $\text{Prob}_Q$($X$ and its entire subsequent tail) is sufficiently small\footnote{Define $\epsilon$ as in traditional proofs to quantify this}. That is to say, given $X$ we can define 
\begin{equation*}
S_X = \{ X' \: | \: \text{Prob}_Q (X') \leq \text{Prob}_Q(X) \}
\end{equation*}We then have that $X$ is too ridiculous to have come from $Q$ if the probability $\text{Prob}_Q(S_X)$ is sufficiently small.

\definition[Confidence Interval] Given $X$, identify all $Q$s it may have originated from. The confidence interval\footnote{It's better to write this as a confidence region as opposed to a confidence interval if we're working in spaces of higher dimensionality than $\mathbb{R}^1$} is defined as $$\{Q \: | \: X \text{ is not too ridiculous to have originated from } Q\}$$

\definition[Hypothesis Test] Given hypothesis $Q$, find all $X$ that will allow for disproving $Q$. The rejection interval (or region) is defined as $$\{X \: | \: X \text{ is too ridiculous to have come from } Q \}$$

\subsection{Two Significant Theorems}

We'll discuss two important theorems that define the notions of $\text{Prob}_Q(X)$ and $\text{Prob}_Q(S)$.

\subsubsection{$\text{Prob}_Q(X)$}
Assume that we are given a source $Q$ that produces observed data outputs $X_1 \dots X_N$ (all abbreviated as the vector $X$) and that the values are identically and independently distributed. We will attempt to define the value $\text{Prob}_Q(X)$. If we were to histogram the vector $X$ (with the y axis representing the frequency of occurrences of $\xi$ in $X$ and the x axis representing the discrete values $\xi_1 \dots \xi_k$)\footnote{$X$ comprises of discrete values that are represented by $\xi$}, each value $\xi_i$ would have an associated frequency $N_i$. Call this histogram $P_X$ with $N_1 + N_2 + \dots + N_k = N$. We then have 
\begin{align*}
\text{Prob}_Q(X) &= Q(X_1) \times Q(X_2) \times \dots \times Q(X_N) \\
	&= Q(\xi_1)^{N_1} \times Q(\xi_2)^{N_2} \times \dots \times Q(\xi_k)^{N_k} \\
	&= 2^{-[N_1 \log \frac{1}{Q(\xi_1)} + N_2 \log \frac{1}{Q(\xi_2)} + \dots + N_k \log \frac{1}{Q(\xi_k)}]} \\
	&= 2^{-N[P_X(\xi_1) \log \frac{1}{Q(\xi_1)} + \dots + P_X(\xi_k) \log \frac{1}{Q(\xi_k)}]} 
\end{align*}
where $Q(X_i)$ is the probability of seeing $X_i$ in the distribution of $Q$. We can multiply each log term by $P_X(\xi_i)$ on the numerator and denominator to express the value as a function of the KL divergence and entropy. We therefore have the following result\footnote{We've only proved the result for a discrete i.i.d distribution, but it can be shown to be applicable to continuous distributions (with differential entropy). This result cannot, however, be extended to non-i.i.d distributions because of the first step}.

\theorem $\text{Prob}_Q(X) = 2^{-N[D(P_X || Q) + H(P_X)]}$

\subsubsection{$\text{Prob}_Q(S)$}

We can begin by writing 
\begin{align*}
\text{Prob}_Q(S) &= \sum_{X \in S} \text{Prob}_Q(X) \\
	&= \sum_{X \in S} 2^{-N[D(P_X || Q) + H(P_X)]}
\end{align*}
With $Q$ representing the true distribution, we have a space of histograms $\{P_X \: \forall \:  X \in S\}$. We want to identify the distribution $P^*$ that is ``closest'' to $Q$. By the Pythagorean inequality (which we'll prove later), we can write the above expression as
$$\text{Prob}_Q(S) \leq \sum_{X \in S} 2^{-N[D(P_X || P^*) + D(P^* || Q) + H(P_X)]}  = 2^{-ND(P^* || Q)} \sum_{X \in S} 2^{-N[D(P_X || P^*) + H(P_X)]}$$which can be written as $$2^{-ND(P^*||Q)} \sum_{X \in S} \text{Prob}_{P^*}(X) = 2^{-ND(P^*||Q)} \text{Prob}_{P^*}(S)$$Since the probability term is less than or equal to one, we've therefore bounded $\text{Prob}_Q(S)$.
\theorem[Sanov's Theorem]  $\text{Prob}_Q(S) \leq 2^{-ND(P^*||Q)}$

\subsubsection{Proof of the Pythagorean Inequality} 
\theorem[Pythagorean Inequality] $D(P||Q) \geq D(P || P^*) + D(P^* || Q)$ where $Q$ is the true distribution and $P$ is the estimated distribution. $P^*$ is the closest distribution in terms of KL divergence.
\begin{proof}
Consider any $P$ in the convex vector space of histograms $S$. By the definition of a convex vector space, $$P_\lambda = \lambda P + (1 - \lambda) P^*$$If we let $\lambda = 0$, $P_\lambda = P^*$, but since we know that $P^*$ is defined to be the minimum of all $P$ in set $S$ (according to the condition $D(P^* || Q) = \min_{P \in S} D(P || Q)$), we know that it is also the minimum of $D(P_\lambda || Q)$ along the path $P^* \rightarrow P$. So, $D_\lambda = D(P_\lambda || Q) = \sum P_\lambda \log P_\lambda / Q$. Therefore, $d D_\lambda / d \lambda$ as a function of $\lambda$ is nonnegative at $\lambda = 0$ because as $\lambda \rightarrow 0$, the divergence decreases. Specifically, $$\frac{d D_\lambda}{d \lambda} = \sum_{\text{bins}} \left[ (P - P^*) + (P - P^*) \log \frac{P_\lambda}{Q} \right]$$At $\lambda = 0$, $P_\lambda = P^*$. We also know that $\sum P(X) = \sum P^*(X) = 1$ by the property of a pdf. We can then write $$\frac{d D_\lambda}{d \lambda} \bigg|_{\lambda = 0} = \sum \left[ P(X) - P^*(X) \right] \log \frac{P^*}{Q}$$This simplifies to $\sum P \log \frac{P^*}{Q} - \sum P^* \log \frac{P^*}{Q} \geq 0$. The first term can be written as $\sum P \log \frac{P^*}{P} \frac{P}{Q} - \sum P^* \log \frac{P^*}{Q}$ which completes the proof (as the initial point $\lambda = 0 \geq 0$ so the remainder of the function is).
\end{proof} 

\subsection{Connections to Statistics}

From the definition of KL divergence, we can write $D[\text{Bernoulli}(p_1) || \text{Bernoulli}(p_2)]$ as $$p_1 \log \frac{p_1}{p_2} + (1-p_1) \log \frac{1-p_1}{1-p_2}$$which is equivalent to\footnote{Converting log to ln} $$\frac{1}{\ln 2} \left[ p_1 \ln \frac{p_1}{p_2} + (1-p_1) \ln \frac{1-p_1}{1-p_2} \right] $$ A second order approximation of $\ln (1 / 1-\epsilon) \approx \epsilon + \epsilon^2 / 2$ allows us to write $$\frac{1}{\ln 2} \left[ p_1 \left( \left( 1 - \frac{p_2}{p_1} \right) + \frac{1}{2} \left( 1 - \frac{p_1}{p_2} \right)^2 \right) + (1-p_1) \left( \left( 1 - \frac{1-p_2}{1-p_1} \right) - \frac{1}{2} \left( 1 - \frac{1 - p_2}{1 - p_1} \right)^2 \right) \right]$$which simplifies to $$ \left( -\frac{1}{2 \ln 2} \right) \frac{(p_1 - p_2)^2}{p_1(1-p_1)}$$What we've just obtained is an approximation to the KL divergence between two binomials. Furthermore, the KL divergence between Gaussian distributions $D[N(\mu_1, \sigma_1^2) || N(\mu_2, \sigma_2^2)]$ is $$\frac{1}{\ln 2} \left[ \ln \frac{\sigma_2}{\sigma_1} + \frac{1}{2} \left( \frac{\sigma_1^2}{\sigma_2^2} - 1 \right) + \frac{(\mu_1 - \mu_2)^2}{2 \sigma_2^2} \right]$$In the special case that $\sigma_1 = \sigma_2 = \sigma$ we have $$\frac{1}{2\ln 2} \frac{(\mu_1 - \mu_2)^2}{\sigma^2}$$Note for future reference that $$D(f || q) = \int_x f \log \frac{f}{q} dx$$

\subsubsection{Connection 1: Observation Studies for Proportion}

Consider a binomial distribution with probability $p$ originating from source $Q$ (that outputs values $x_1, x_2, \dots, x_N$). We've defined $X$ is too ridiculous to have come from $Q$ as $2^{-ND(P^*||Q)} < \epsilon$\ (by extension of Sanov's theorem). But since $P^*$ (our best guess of the distribution) is equivalent to Binomial($\hat{p}$)\footnote{That is, we make our best guess given the observed probability distribution}, we can write $$2^{-N \left[ \frac{1}{2 \ln 2} \frac{(\hat{p} - p)^2}{\hat{p}(1 - \hat{p})} \right]} < \epsilon$$Which is equivalent to $$ \frac{1}{2 \ln 2} \frac{(\hat{p} - p)^2}{(\hat{p}(1 - \hat{p}))/N} > \log \frac{1}{\epsilon}$$ We can rewrite this as $$\frac{(\hat{p} - p)^2}{(\hat{p}(1 - \hat{p}))/N} > \ln \frac{1}{\epsilon^2}$$ Taking the square root of both sides, we have $$\frac{|\hat{p} - p|}{ \sqrt{\frac{\hat{p}(1 - \hat{p})}{N}}} > \underbrace{\sqrt{\ln \frac{1}{\epsilon^2}}}_{\text{call this constant } C}$$ We can now express this in a more familiar form. In particular, given $X$ = \{$P$ | $P$ lies in the interval $\hat{p} \pm C \sqrt{\frac{\hat{p}(1 - \hat{p})}{N}}$ \}, then we have the hypothesis test that $\hat{p}$ is ``too ridiculous'' to have come from $X$ if it lies in the rejection region \{ $\frac{|\hat{p} - p|}{ \sqrt{\frac{\hat{p}(1 - \hat{p})}{N}}} > C$ \}. 

\subsubsection{Connection 2: Observation Studies for Mean}

Consider a normal distribution with mean $\mu$ and standard deviation $\sigma$ originating from source $Q$ (that outputs values $x_1, x_2, \dots, x_N$). We've defined $X$ is too ridiculous to have come from $Q$ as $2^{-ND(P^*||Q)} < \epsilon$\ (by extension of Sanov's theorem). But since $P^*$ (our best guess of the distribution) is equivalent to $N(\bar{x}, \sigma^2)$, we can write $$2^{-N \left[ \frac{1}{2\ln 2} \frac{(\bar{x} - \mu)^2}{\sigma^2} \right]} < \epsilon$$Which is equivalent to $$\frac{(x - \mu)^2}{\sigma^2/N} > 2 \ln 2 \log \frac{1}{\epsilon}$$Taking the square root of both sides, we have $$\frac{|\bar{x} - \mu|}{\sigma / \sqrt{N}} > \underbrace{\sqrt{\ln \frac{1}{\epsilon^2}}}_{\text{call this constant } C} $$We can now express this in a more familiar form. In particular, the confidence interval given $X$ from distribution $Q$ is $\{  \mu | \frac{|\bar{x} - \mu|}{\sigma / \sqrt{N}} \leq C \} = \{ \mu \: | \: \mu \text{ lies in the interval } \bar{x} \pm C \sigma / \sqrt{N} \}$

\subsubsection{Connection 3: Experimental Studies (General)}

In an experimental study, we have inputs $x_1, x_2, \dots, x_N$ which pass through a channel $Q(y|x)$ to produce outputs $y_1, y_2, \dots, y_N$. Our job is to guess $Q(y|x)$ or refute $\hat{P}(y|x)$ that has been guessed by someone else. Given that $X$ went into $Q$, $Y$ is too ridiculous to have come out of $Q$ by a certain criterion. To define this criterion, assume that $X = \left< x_1, x_2, \dots, x_N \right>$ can assume possible values $\xi_1, \xi_2, \dots, \xi_k$. Now we segregate $Y$ according to $\xi_1, \xi_2, \dots, \xi_k$. The vector $Y$ is split into $y_{\xi_1}, y_{\xi_2}, \dots, y_{\xi_k}$ where each $y_{\xi_i}$ represents the components of $Y$ where the corresponding $x$ have values equal to $\xi_i$. We can next segregate the true conditional distribution $Q(y|x)$ according to $\xi_1, \xi_2, \dots, \xi_k$ (that is, $Q_{\xi_i}(y) = Q(y | x = \xi_i)$). We define multiple tails, one for each input value, where $$\text{tail of } y_{\xi_i} = \{ y_{\xi_i}' \: | \: \text{Prob}_{Q_{\xi_i}}(\text{empirical histogram of } y'_{\xi_i}) \leq \text{Prob}_{Q_{\xi_i}} (\text{empirical histogram of } y_{\xi_i}) \}$$Given that $X$ went into $Q$ and $Y$ is too ridiculous to have come out of $Q$, we can write $$\text{Prob}_{Q_{\xi_1}}(\text{tail of } y_{\xi_1}) \times \text{Prob}_{Q_{\xi_2}}(\text{tail of } y_{\xi_2}) \times \dots \times \text{Prob}_{Q_{\xi_k}}(\text{tail of } y_{\xi_k}) < \epsilon $$ Recalling our earlier definition of Prob$_Q$(tail of $X$) $\leq 2^{-ND(P^*||Q)}$, we have that $P^*$ is the closest histogram to $Q$ among all the histograms you can draw of things in the tail of $X$. We can thus replace each of the probability terms with $2^{-N_iD(P_{\xi_i}^*||Q_{\xi_i})} \: | \: i \in 1 \dots k$.

\subsubsection{Connection 3a: Experimental Studies for Proportion}

In this case, we have $Q(y|x) \rightarrow Q_{\xi_1}(y) \sim \text{Binomial}(p_1)$ and $Q(y|x) \rightarrow Q_{\xi_2}(y) \sim \text{Binomial}(p_2)$. We can follow the same derivation as the previous steps (but keeping in mind the $N_i$ terms) to obtain $$\frac{(\hat{p_1} - p_1)^2}{(\hat{p_1}(1 - \hat{p_1}))/N_1} + \frac{(\hat{p_2} - p_2)^2}{(\hat{p_2}(1 - \hat{p_2}))/N_2} > \ln \frac{1}{\epsilon^2}$$Moving the $\ln 1/ \epsilon^2$ to the other side, we have an equation for an ellipse (if we graph with respect to the variables $p_1$ and $p_2$). All non rejected proportions are within the area of this ellipse, which is centered at $(\hat{p_1}, \hat{p_2})$ and has a horizontal axis radius of $\sqrt{\ln \frac{1}{\epsilon^2} \frac{\hat{p_1}(1 - \hat{p_1})}{N_1}}$. We can obtain a confidence interval by setting $p_1 = p_2$, which allows us to draw two tangents to the ellipse. We can find the intersection points of the tangents with the ellipse, and the distance between these points defines the confidence interval. 

To identify the confidence interval, consider two tangents $y - x = C_1$ and $y - x = C_2$. We need to find $C_2 - C_1$. Consider the equation of an ellipse $$\frac{(x-h)^2}{a^2} + \frac{(y-k)^2}{b^2} = 1$$Implicitly differentiating both sides yields $$\frac{dy}{dx} = -\frac{b^2}{a^2} \left( \frac{x-h}{y-k} \right)$$which we can equate to 1 to obtain $$x-h = \pm \sqrt{\frac{a^4}{b^2 + a^2}}$$Plugging this into the ellipse equation yields $$y - k = \mp \frac{b^2}{a^2} (x-h)$$We therefore have two pairs of $x$ and $y$ (one where $x$ is $+$ and $y$ is $-$, and vice-versa). We can then subtract these two to obtain $C_1 - C_2 = 2 \sqrt{a^2 + b^2}$. Expressing this int terms of proportions yields the confidence interval $$(\hat{p_1} - \hat{p_2}) \pm C \sqrt{ \frac{\hat{p_1} ( 1 - \hat{p_1})}{N_1} + \frac{\hat{p_2} (1 - \hat{p_2})}{N_2}}$$

\section{Unit 3: Kolmogorov Complexity}

So far we have talked about descriptive complexity and related information theory to pdfs. Kolmogorov Complexity is a measure of algorithmic complexity, which consists of a set of subjective statements that a human or a compiler interprets. Kolmogorov Complexity is independent of compiler choice. 

\subsection{Turing Machine}

A Turing machine is a device that converts input to output using a set of predefined rules. An example of a basic Turing machine is an FSM which receives input one bit at a time, and given its current state and input, writes something to the output or write something in its "work tape," or internal memory. The only stipulation is that the input has to be processed sequentially (right to left). 

\subsection{Definitions}

\begin{align*}
	x: \text{finite length binary string} \\
	U: \text{universal computer} \\
	l(x): \text{length of string } x \\
	U(p): \text{output of computer} U \text{when presented with program } p 
\end{align*}

\definition{Kolmogorov Complexity} Kolmogorov complexity $K_u(x)$ of a string $x$ with respect to $U$ is $K_U(x) = \min_{p: U(p)=x} l(p)$, that is, the length of the shortest program which produces output $x$ when fed into universal computer $U$.

\subsection{Examples}

\example The description of string $010101010101...01$ is simply an alternating pattern of 0s and 1s.
\example The description of string $0110101000001001111001101100...00001000$ is the binary expansion of $\sqrt{2}-1$
\example Describing pseudo-random string $110111100111010111110110111110...100111011$ is slightly more complex. Let $k$ be the number of 1s in the sequence. A table with all possible sequences with $k$ 1s can be precomputed and stored in the universal computer. The index of the particular sequence can be precomputed and passed as the input of the program. Using this strategy, the minimum program length is $log n + n H\frac{k}{n}$.

\subsection{A Theorem}

\theorem{Universality of K.C.} If $U$ is a universal computer, then for any other computer, $A$, 
\begin{align*}
	K_u(x) = K_A(x) + C_A
\end{align*}

for all strings $x \in {0, 1}$ where $C_A$ does not depend on $x$.

\subsubsection{Proof}

Assume we have $P_A$ for computer $A$ to print $x$.
\begin{itemize}
	\item $A(P_A) = x$
	\item precede this program by a simulation program $S_A$ which tells $U$ to simulate $A$.
	\item computer $U$ will then interpret instructions in the program for $A$, perform calc., print $x$.
	\item program for $U: p=S_Ap_A$ with length $l(p)=l(S_A)+l(p_A)=C_A+l(p_A)$
	\item $K_U(x)=\min{l(p)} = \min_{A(p)=x}{l(p_A)+C_A}=K_A(x)+C_A$
	\item When the program size is large, the constant is deemed irrelevant, and from now on, we'll assume the Turing machine is universal.
\end{itemize}

\subsection{Another Theorem}

\theorem{} $K(x|l(x)) \leq l(x) + c$ \\

The reasoning is that if we know $l(x)$, the end of the program is clearly defined. A program for printing $x$ is "Print the following $l$-bit sequence: $x_1x_2x_3x_{l(x)}$.

If $l(x)$ is not known, we need to find a way to let the Turing machine know that we have reached the end of the program. A simple way to do this is to add an additional symbol so that $K(x) \leq K(x|l(x))+2\log{l(x)}+c$.

\subsubsection{Proof}

Assume we have a Turing machine that accepts $01$ as a comma. Let $l(x)$ be $n$. To describe $l(x)$ repeat every bit of $n$ twice, then end with $01$. If the length is 5, we pass $11001101$. The length of the description of $l(x)$ using this method is $2\log{n}+2$ bits. Another way to do this would be using recursion.

\subsection{Yet Another Theorem}
\theorem{} Number of strings $x$ with complexity $K(x) < k$ satisfies $|{x \in {0, 1}^*: K(x) < k}| < 2^k$.
\subsubsection{Yet Another Proof}
There are $2^k-1 < 2^k$ sequences with length $< k$.

\subsection{3/13/2017 (Rajiv)}

\subsubsection{Some Entropy Notation}

Let's start by introducing some "new" definitions for entropy. We know that $H(p)$ is given by $$H_0(p) = -p\log p - (1-p)\log(1-p).$$ Along the same lines, let's write the entropy of a mean of values $x_i$ as follows: $$H_0\left(\frac{1}{n}\sum x_i\right) = -\overline{x}_n \log \overline{x}_n - (1-\overline{x}_n)\log(1-\overline{x}_n).$$ This shouldn't be confused with the entropy of $\overline{x}$ across many trials; it's the entropy of $\operatorname{Bern}(p = \overline{x}_n)$.

\subsubsection{Lemma}

As an intermediate step in another proof, we'd like to show that $$\binom{n}{k} \le 2^{nH_0(\frac{k}{n})}.$$ Stirling's approximation formula will help here: $$n! = \sqrt{2\pi n} \left(\frac{n}{e}\right)^n.$$

\textbf{Proof.} We'll write $\binom{n}{k} = \frac{n!}{k!(n-k)!}$ and use Stirling's approximation:
\begin{align*}
\ln \binom{n}{k} &= \frac{1}{2} \ln (2\pi) + \frac{1}{2} \ln n + n \ln n - n \\
&-\left[ \frac12 \ln(2\pi) + \frac12 \ln k + k \ln k - k \right] \\
&-\left[\frac12 \ln(2\pi) + \frac12 \ln(n-k) + (n-k)\ln(n-k) - (n-k)\right] \\
\frac 1n \ln \binom nk &= \frac{1}{2n}\left(\ln\left[\frac{n}{k(n-k)}\right]\right) + \ln n - \left(\frac kn\right) \ln k - \left(\frac{n-k}{n}\right) \ln(n-k) - \frac{1}{2n}\ln(2\pi) \\
&= -\frac{k}{n}\ln\left(\frac kn\right) - \left(1 - \frac kn\right) \ln \left(1 - \frac kn\right) \\
&= H_0\left(\frac kn\right).
%\frac12\ln\left(\frac{n}{k(n-k)}\right) - n\left[\left(\frac kn\right) \ln \left(\frac kn\right) - \left(\frac{n-k}{n}\right)\ln\left(\frac{n-k}{n}\right)\right] - \ln(2\pi).
\end{align*}

This above derivation isn't perfect, but the idea is that we're using Stirling's approximation to show this Lemma -- nbd if the algebra above doesn't \textit{entirely} work out.

\subsubsection{Basic Kolmogorov Complexity Examples}

Let's look at some concrete examples of $K_U(x)$. For a string of repeating zeros, the number of bits in the program is constant. Same for some other repeating pattern.

\example The Mandelbrot set is a particular set of complex numbers that produces nice looking fractal boundaries. Turns out that the Kolmogorov complexity of this is also constant, very close to zero, showing how theoretical or visual complexity does not necessarily correlate with $K_U(x)$.

\example If we have an integer $n$, $K(n) \le \log n + c$, since it requires $\log n$ bits to represent the integer.

\example Sequence of $n$ bits where $k$ of the bits are ones. Here, $k$ can go from $0$ to $n$, and another counter index $i$ can go from $0$ to $\binom nk$. So the length of the program is $\ell(p) = \log k + \log \binom nk + c$. Then, we can apply our previous Lemma for a new theorem!

\theorem $K(x_1,x_2,\dots,x_n \mid n) \le nH_0\left(\frac kn\right) + \log k + c$.
\\ \\
\noindent Note: we can discuss the Kolmogorov complexity of integers in the same way of binary strings, because of example 11. For some integer $n$, $$K(n) = \min_{p : U(p) > n} \ell(p).$$

\subsubsection{Kolmogorov Complexity and Entropy}

The expected value of a random sequence is close to Shannon entropy. Specifically, we'll prove that it satisfies a similar bound to Huffman. Just as we used the Kraft inequality to prove Huffman, we'll use the Kraft inequality to prove this.

\theorem For any computer $U$, $$\sum_{p : U(p) \text{ halts}} 2^{-\ell(p)} \le 1.$$

\newpage
\section*{Appendix A---Quotes}
\begin{itemize}
\item ``I have a problem. It's called gambling.'' (Dr. Aiyer)
\item ``What's $E$? Entropy?'' (David Zhu)
\item ``So is it the strong law of weak numbers?'' (Jerry Chen)
\item ``It's like a half life... but it's a double life'' (Steven Cao)
\item ``Isn't this just Lagrange?''  \\ 10 minutes later.. \\ ``Wait, how do you do Lagrange again?'' (Swapnil Garg)
\item ``I'm just amazed that you manage to learn something'' (Dr. Aiyer)
\item (Looking at $\Sigma$) That's a backwards $\xi$! (Steven)
\item ``What's a binomial distribution?'' (Dr. Aiyer) \\ ``$a + bx$'' (Shaya)
\item ``An approximation of $\ln(1-x)$ is $1 + x + x^2 + x^3 + \dots$'' (Misha Ivkov)
\item ``Where does $\epsilon$ go to get a haircut?'' \\ ``The epSALON'' (Shaya)
\item ``So the sample mean is $\hat{\mu}$...'' (Manan Shah)
\item ``Yes, $e^{1/2}$ on most days is $\sqrt{e}$...'' (Dr. Aiyer)
\item ``We're going to try to finish this today (Ha!)'' (Dr. Aiyer)
\item ``Why are we using a $\times$ for multiplication in 2017?'' \\ ``I thought we were at the age where we didn't have to use anything...'' (Rajiv Movva)
\item ``Why do we discuss confidence intervals instead of regions in statistics?'' (Dr. Aiyer) \\ ``We approximate the ellipse to a square'' (Shaya)
\end{itemize}
\end{document}
